\documentclass[modern]{aastex62}

\usepackage{stix}
\newcommand{\vdag}{(v)^\dagger}
\newcommand\aastex{AAS\TeX}
\newcommand\latex{La\TeX}

\graphicspath{{./}{figures/}}
\shorttitle{HD106906b Time Resolved Observations}
\shortauthors{Zhou et al.}

\begin{document}

\title{HD106906 Working Papers}

\correspondingauthor{Yifan Zhou}
\email{yzhou@as.arizona.edu}

\author{Yifan Zhou}
\affil{Steward Observatory}

\begin{abstract}
  This documents keep the main result for \emph{Cloud Atlas} HD106906b \citep{Bailey2013} observations.
\end{abstract}

\keywords{}

\section{Introduction}
HD106906b is a mid-L type planetary mass companion \citep{Bailey2013}.

\section{Observations}
We observed HD106906b on DDMMYY for two consecutive HST orbits as part of the variability amplitude assessment survey and on DDMMYY2 for seven consecutive HST orbits as part of the Deep Look survey. 

\section{Data Reductions}
% include the figures here
\subsection{Two Roll Differential Imaging}
\subsection{Ramp Correction}
\subsection{Astrometry}

\section{Results}
\subsection{Image}
\subsection{Light Curve}
\subsection{Astrometry}

\section{Discussion}
\begin{enumerate}
\item the variability
\item SED for HD106906, further determine its spectral type
\item the limit on the inclination \citep[see][]{Vos2017}
\item possible astrometry constrains (what about the distortion correction for WFC3)
\item limit on additional companions
\item lack of large amplitude detection for planetary mass companions. 
\end{enumerate}
For the last point, take the variability occurance rate from \citep{Vos2017,Metchev2015}, test whether the low mass companions' variability occurance rate agree with the low surface gravity field objects

\bibliographystyle{yahapj}
\bibliography{library}

\end{document}
%%% Local Variables:
%%% mode: latex
%%% TeX-master: t
%%% End:
