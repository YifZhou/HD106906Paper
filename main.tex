\documentclass[twocolumn, trackchanges]{aastex62}

\usepackage[all]{nowidow}
\usepackage{stix}
\usepackage{xspace}
\usepackage{hyperref}
\usepackage{todonotes}
\usepackage{graphicx}

\newcommand{\vdag}{(v)^\dagger}
\newcommand\aastex{AAS\TeX}
\newcommand\latex{La\TeX}
\newcommand{\mjup}{\ensuremath{M_\mathrm{Jup}}\xspace}
\newcommand{\teff}{\ensuremath{T_{\mathrm{eff}}}\xspace}
\newcommand{\logg}{\ensuremath{\log(g)}\xspace}
\newcommand{\fluxunit}{\ensuremath{\mathrm{ergs}\,\mathrm{cm^{-2}}\,\mathrm{s^{-1}}\,\mathrm{\micron^{{-1}}}}}
\graphicspath{{./}{figures/}}
\shorttitle{HD106906b Time-Resolved Observations}
\shortauthors{Zhou et al.}
\newcommand{\editHL}[1]{{\color{blue}#1}}
\begin{document}


\title{Cloud Atlas: High-precision HST/WFC3/IR Time-Resolved Observations of Directly-Imaged Exoplanet HD106906b}

\correspondingauthor{Yifan Zhou}
\email{yzhou@as.arizona.edu}

\author{Yifan Zhou}
\affiliation{Department of Astronomy/Steward Observatory, The University of Arizona, 933 N. Cherry Avenue, Tucson, AZ, 85721, USA}
\affiliation{Department of Astronomy/McDonald Observatory, The University of Texas, 2515 Speedway, Austin, TX, 78712, USA}

\author{D\'aniel Apai}
\affiliation{Department of Astronomy/Steward Observatory, The University of Arizona, 933 N. Cherry Avenue, Tucson, AZ, 85721, USA}
\affiliation{Department of Planetary Science/Lunar and Planetary Laboratory, The University of Arizona, 1640 E. University Boulevard, Tucson, AZ, 85718, USA}
\affiliation{Earths in Other Solar Systems Team, NASA Nexus for Exoplanet System Science.}

\author{Luigi R. Bedin}
\affiliation{INAF – Osservatorio Astronomico di Padova, Vicolo dell'Osservatorio 5, I-35122 Padova, Italy}

\author{Ben W. P. Lew}
% \affiliation{Department of Astronomy/Steward Observatory, The University of Arizona, 933 N. Cherry Avenue, Tucson, AZ, 85721, USA}
\affiliation{Department of Planetary Science/Lunar and Planetary Laboratory, The University of Arizona, 1640 E. University Boulevard, Tucson, AZ, 85718, USA}

\author{Glenn Schneider}
\affiliation{Department of Astronomy/Steward Observatory, The University of Arizona, 933 N. Cherry Avenue, Tucson, AZ, 85721, USA}


\author{Elena Manjavacas}
\affiliation{Department of Astronomy/Steward Observatory, The University of Arizona, 933 N. Cherry Avenue, Tucson, AZ, 85721, USA}


\author{Theodora Karalidi}
\affiliation{Department of Astronomy and Astrophysics, University of California Santa Cruz, 1156 High Street, Santa Cruz, CA 95064, USA}

\author{Stanimir Metchev}
\affiliation{Department of Physics \& Astronomy and Centre for Planetary Science and Exploration, The University of Western Ontario, London, Ontario N6A 3K7, Canada}
\affiliation{Department of Physics \& Astronomy, Stony Brook University, 100 Nicolls Rd, Stony Brook, NY 11794-3800, USA}


\author{Paulo A. Miles-P\'aez}
\affiliation{Department of Physics \& Astronomy and Centre for Planetary Science and Exploration, The University of Western Ontario, London, Ontario N6A 3K7, Canada}
\affiliation{Department of Astronomy/Steward Observatory, The University of Arizona, 933 N. Cherry Avenue, Tucson, AZ, 85721, USA}

% \author{Nicolas B. Cowan}
% \affiliation{Department of Earth \& Planetary Sciences and Department of Physics, McGill University, 3550 Rue University, Montr\'eal, Quebec H3A 0E8, Canada}

% \author{Mark S. Marley}
% \affiliation{NASA Ames Research Center, Mail Stop 245-3, Moffett Field, CA 94035, USA}

% \author{Jacqueline Radigan} \affiliation{Utah Valley University, 800 West University Parkway, Orem, UT 84058, USA}



% \author{Patrick J. Lowrance}
% \affiliation{IPAC-Spitzer, MC 314-6, California Institute of Technology, Pasadena, CA 91125, USA}


% \author{Adam J. Burgasser} \affiliation{Center for Astrophysics and Space Science, University of California San Diego, La Jolla, CA 92093, USA}


\begin{abstract}
 HD106906b is an $\sim11\mjup$, $\sim 15$\,Myr old directly-imaged exoplanet orbiting at an extremely large distance from its host star. The wide separation between HD106906b and its host star greatly reduces the difficulty in direct-imaging observations, making it one of the most favorable directly-imaged exoplanets for detailed characterization. In this paper, we present HST/WFC3/IR time-resolved observations of HD106906b in the F127M, F139M, and F153M bands. We have achieved $\sim1\%$ precision in the lightcurves in all three bands. The F127M lightcurve demonstrates marginally-detectable variability with a best-fitting period of 4.02 hr, while the lightcurves in the other two bands are consistent with flat lines. We construct primary-subtracted deep images and use these images to exclude additional companions to HD106906 that are more massive than 2\mjup{} and locate at projected distances for more than $700$ au. We measure the astrometry of HD106906b in two HST/WFC3 epochs and achieve precisions better than 2.5 mas. The position angle and separation measurements do not deviate from those in the 2004 HST/ACS/HRC images for more than 1$\sigma$ uncertainty. We provide the HST/WFC3 astrometric results for 25 background stars that can be used as reference sources in future precision astrometry studies. Our observations also provide the first 1.4 \micron{} water band photometric measurement for HD106906b. Fitting HD106906b's spectral energy distribution to the BT Settl model shows that the model over-predicts the HD106906b's water absorption depth and the best-fitting value for \logg{} is inconsistent with previous low to intermediate surface gravity assessment. These inconsistencies highlight the challenges in modeling atmospheres of young planetary-mass objects.
\end{abstract}

\keywords{Planetary Systems --- planets and satellites: atmospheres --- methods: observational}

\section{Introduction}

Condensates clouds are central components of the atmospheres of brown dwarfs and exoplanets \citep[e.g.,][]{Morley2012,Marley2013,Marley2015}. Cloud opacity strongly impacts the near-infrared color and spectra of these objects. Therefore, understanding cloud properties is critical to determining  fundamental properties and atmospheric compositions of substellar objects through emission and transmission spectroscopic observations \citep[e.g.,][]{Ingraham2014, Kreidberg2014a, Stevenson2016, DeWit2016, Samland2017}. Because brown dwarfs are available for direct spectroscopy and their observations are generally less challenging than spectroscopic observations of transiting exoplanets, cloud properties for brown dwarfs are observationally tighter constrained than for exoplanets through time-averaged spectroscopic \citep[e.g.,][]{Cushing2008,Stephens2009} and {time-resolved} \citep[e.g.,][]{Buenzli2012,Apai2013,Yang2016,Biller2017,Apai2017} observations. Directly-imaged exoplanets and planetary-mass companions \citep[e.g.,][]{Chauvin2004,Marois2008a,Marois2010,Macintosh2015a}, which overlap with transiting planets in mass and are suitable for high-quality time-series observations, are excellent targets for connecting condensate cloud studies of brown dwarfs and exoplanets.

HD106906b is an $11\pm2$ \mjup{}  mass exoplanet orbiting an F5V spectral-type star \citep{Bailey2013}. \added{Based on spectroscopic analysis \citep{Bailey2013,Daemgen2017}, the planet has an effective temperature (\teff) of 1800~K and a spectral type classification of L2.5-3.} The HD106906 system, at a distance of $103.3\pm0.4$ pc \citep{Gaia2016,Gaia2018}, is a member of the Lower Centaurus Crux association (99.8\% membership probability based on BANYAN-$\Sigma$, \citealt{Gagne2018} ), which itself is  part of the Sco-Cen OB association. \added{Based on the cluster membership classification, the age of the system is  $15\pm3$ Myr \citep[][]{Pecaut2016}.} The planet has a wide separation of $7\arcsec.11\pm0\arcsec.03$ from its host star \citep{Bailey2013}, corresponding to a projected distance of $734\pm4$ au. \replaced{Because of its wide angular separation to its host and moderate brightness contrast ($\Delta J=10.3\,\mathrm{mag}$),}{Because of the system's large angular separation, incident flux from the bright host star does not contaminate that from the companion significantly, despite the large brightness contrast ($\Delta J=10.3\,\mathrm{mag}$). Therefore,} HD106906b is among the most favorable exoplanets for atmospheric characterization \citep[e.g., ][]{Bailey2013,Kalas2015,Wu2016,Daemgen2017}.

Multi-wavelength photometric \citep{Bailey2013,Kalas2015,Wu2016} and spectroscopic \citep{Bailey2013, Daemgen2017} observations have been used to characterize HD106906b's atmosphere.
\deleted{These studies agree on an effective temperature (\teff) of 1800~K and a spectral type classification of L2.5-3. Based on its ``triangular-shaped'' $H$-band spectrum, both \citet{Bailey2013} and \citet{Daemgen2017} classified HD106906b as an intermediate to low surface gravity object. The low surface gravity classification is also consistent with its young age.} \explain{These sentences are not directly relevant to the key point of this paragraph: time-resolved observations are an effective tool for atmospheric characterization of HD106906. So I deleted the above sentences.} In these observations, similar to many other young L-type planetary-mass objects (2M1207b, HR8799bcde, PSO J318), HD106906b appears reddened near-infrared (NIR) colors compared to those of the field brown dwarfs of the same spectral type. The reddened NIR color is often associated with dusty atmospheres and thick condensate clouds \citep[e.g.,][]{Skemer2011,Bowler2013,Liu2016}. Time-resolved observations of these reddened objects have often found  them to be variable \citep[e.g.,][]{Biller2015,Zhou2016,Lew2016,Vos2017,Biller2017,Manjavacas2017,Zhou2019}. The most likely explanation for their variability is heterogeneous clouds rotationally modulating the integrated flux from HD106906b's photosphere. Consequently, multi-wavelength NIR rotational modulation became an effective tool to study condensate clouds, particular vertical cloud profiles for brown dwarfs and planetary mass objects \citep[e.g.,][]{Apai2013, Biller2017,Manjavacas2017,Manjavacas2019,Paez2019,Zhou2019}. Therefore,  high-precision time-resolved NIR observations can also be an effective method to explore the cloud properties of HD106906.

HD106906b's extremely wide orbit and its deviation from the host star's circumstellar plane pose challenges in explaining its formation \citep{Bailey2013,Kalas2015,Wu2016}. Disk fragmentation has difficulty forming a planet/companion with a mass as small as that of HD106906b \citep[e.g.,][]{Kratter2010}. High-contrast direct-imaging survey results strongly support core accretion as the formation pathway of planetary-mass companions with orbits smaller than 100~au \citep{Wagner2019,Nielsen2019}.  At a projected distance of more than 700~au from its host star \citep{Bailey2013}, it is unlikely for HD106906b to accrete enough material through \emph{in situ} core accretion. A $\sim21^{\circ}$ projected angle between the planet's position angle and the plane of its host star's disk \citep{Kalas2015} further argues against \emph{in situ} core accretion but suggests dynamical orbit evolution of this planet \citep[e.g.,][]{Marleau2019}.  \citet{DeRosa2019} discovered a close, near-coplanar stellar encounter with the HD106906 system, further supporting a conjecture of intense dynamical activity in the system's evolution history. Considering these evidence suggesting past dynamical evolution, it should not be surprising if HD106906b has an eccentric orbit. Therefore, astrometric constraints on the orbit of HD106906b will be critical for understanding the formation and evolution history of HD106906b.

In this paper, we analyze and discuss \emph{Hubble Space Telescope} Wide Field Camera 3 near-infrared channel (HST/WFC3/IR) observations of HD106906b in time-resolved direct-imaging mode. We present lightcurves of HD106906b in three bands that cover the 1.4\micron{} water band and its the continuum. We look for variability in the lightcurves and use them to discuss the atmospheric and cloud properties of HD106906b. We also compare the relative astrometry of HD106906 system in the two WFC3 observations and in the HST Advanced Camera for Survey/High-Resolution Channel (ACS/HRC) observations, which were taken in 2004. The WFC3 and ACS/HRC observations together form a high astrometric precision  image series with 14 years baseline,  which can place tight constraints on the relative motion of HD106906b relative to its host star.

\section{Observations}

\begin{figure*}
  \centering
\plottwo{figures/F153M_medianSubtractionMasked_rotation_01}{figures/HD106906_RGB_composite}  
  \caption{Direct-imaging observations of the HD106906 system. \emph{Left:} A demonstration of the two-roll differential imaging results. Red color represents signals from the original images and blue colored pixels are structures from the subtraction model images. Regions that are marked by hatches are used for optimizing the subtraction. \emph{Right: An R (F153M) G (F139M) B (F127M) color composite image of HD106906.} Overlaid on the HST RGB composite are the false-color Gemini Planet Imager (inner most) and ACS/HRC (outer annulus) scattered light images \citep{Kalas2015} of the circumstellar disk. The circumstellar disk is not visible in the WFC3/IR images.}
  \label{fig:2rdi}
\end{figure*}


The HST/WFC3/IR observations of HD106906 are part of the HST Large Treasury program \emph{Cloud Atlas} (Program ID: 14241, PI: D. Apai). We observed HD106906 from 2016-01-29 20:45 to 2016-01-29 23:02 UTC for two consecutive HST orbits as part of the program's variability amplitude assessment survey (VAAS). We then used the same instrument set-ups to re-visit the target from 2018-06-07 02:14 to 2018-06-07 12:35 UTC  for seven consecutive HST orbits as part of the deep look observations (DLO). Dithering was not applied during the observation to reduce systematics caused by flat field errors. The target was observed in F127M ($\lambda_{\mathrm{pivot}}=1.274\micron$, $\mathrm{FWHM}=0.07\micron$), F139M ($\lambda_{\mathrm{pivot}}=1.384\micron$, $\mathrm{FWHM}=0.07\micron$) and F153M ($\lambda_{\mathrm{pivot}}=1.533\micron$, $\mathrm{FWHM}=0.07\micron$) filters.  The filter selection allows comparison of the modulations  in (F139M) and out  (F127M, F153M) of the 1.4 \micron{} water absorption band.  Exposure times were 66.4 seconds for the F127M and F153M observations and 88.4 seconds for the F139M observations. We alternated these three filters in every two or three exposures, and thus the lightcurves in the three filters are almost contemporaneous. 

The observations are designed to enable two-roll angular differential imaging for primary point spread function (PSF) star subtraction.  This technique was successfuly applied in HST high-contrast observations \citep[e.g.,][]{Zhou2016,Zhou2019,Paez2019}. {Successive orbits alternately differed in celestial orientation angle 31 degrees apart, with odd (1, 3, 5, and 7), and even (2, 4, 6) numbered orbits respectively at the same orientations.} Subtracting images  taken in the odd orbits from those  taken in the even orbits (or vice versa) removes the primary star PSF (in the absence of systemmatics to the level of the photon noise) but conserves the companion PSF (Figure \ref{fig:2rdi}).  

HD106906 was also observed by HST/ACS/HRC on 2004-12-01 UTC (PID: 10330, PI: H. Ford). The 2004 ACS/HRC observations include two identical 1,250 seconds direct-imaging exposures in the ACS F606W  band. We use results  from these observations \citep{Bailey2013,Kalas2015} to extend the temporal baseline for our astrometric analysis.

\section{Data Reduction}

\subsection{Time-Resolved Photometry}
We start our time-resolved photometry with the \texttt{flt} files produced by the CALWFC3 pipeline. Our photometric data reduction has four steps: data preparation, primary star subtraction, PSF-fitting photometry, and lightcurve systematics removal. Reductions for lightcurves in the three filters are independent. Therefore, the four reduction steps are applied to observations in the three filters in parallel. 

In the data preparation step, we organize the bad-pixel-masked and sky-subtracted images into data cubes. First, we make bad pixel masks and remove the sky background. Pixels that have data quality flags 4 (bad detector pixel), 16 (hot pixel), 32 (unstable response), and 256 (full-well saturation) are identified as ``bad pixels'', masked out, and excluded from subsequent analyses. We then further examine images by eye to identify and mask out remaining spurious pixels. To remove the sky background, we first draw circular masks around all visible point sources in the field of view and then apply a five-iteration  sigma-clip (threshold: $5\sigma$) to exclude remaining bright pixels. We take the median value of the unmasked pixels as sky background and subtract it from every image. The background-subtracted images and the associated bad pixel masks are sorted in chronological order and stored in data cubes.

We then apply two-roll differential imaging \citep[2RDI, e.g.,][]{Lowrance1999,Song2006} to subtract the PSF of the primary star.  Images taken with the first telescope roll are subtraction template candidates for images taken with the second telescope roll and vice versa. We measure the primary star positional offset in each image using two-dimensional cross-correlation and align the primary star PSFs with bi-linear interpolation shift. We refine image registration by least $\chi^{2}$ optimization in the diffraction spike regions that are caused by the secondary mirror support structures. We then select the best subtraction template from all available candidate images.  Each subtraction template candidate is linearly scaled to minimize the squared summed subtraction residuals in the original$-$template image in an annulus around HD106906A (Figure \ref{fig:2rdi}). The best subtraction template is the one that results in the least subtraction residuals. Finally, we subtract the best templates from the original images to obtain primary subtracted images (Figure \ref{fig:2rdi}). 

HD106906b's flux intensity is measured by PSF fitting to the primary subtracted images. Details of the PSF fitting procedures can be found in \citet{Zhou2019}. We construct $9\times$ over-sampled PSFs using the TinyTim PSF modeling software \citep{Krist1995}. Free parameters for the model PSFs are the centroid coordinates, HST secondary mirror displacement, and the amplitude of the PSF. We optimize these parameters using a maximum likelihood method combined with Markov Chain Monte Carlo (MCMC) algorithms \citep[MCMC performed by \texttt{emcee},][]{Foreman-Mackey2012}. Aperture correction for each filter band is done through PSF fitting photometry as we normalize the model PSF to flux within an infinitely large aperture. 

Finally, we correct the lightcurve systematics and estimate the photometric uncertainty.  For WFC3/IR lightcurves, charge trapping related ramp effect is the major component of lightcurve systematic noise. We use RECTE \citep{Zhou2017} to model and remove the ramp effect systematics from the lightcurves. Our implementation of the ramp effect removal procedure follows \citet{Zhou2019}, in which details of the application of RECTE in time-resolved direct imaging observations are provided. We calculate ramp profiles by feeding the entire time series into RECTE and forward-modeling the charge trapping systematics. The model ramp profiles are divided from the lightcurves to correct the systematics.  We estimate the photometric uncertainty by combining  photon noise, detector readout noise, and dark current in pixels that are used for the measurements. Figure~\ref{fig:lightcurve} shows the corrected lightcurves and uncertainties.

\begin{figure}[!h]
  \centering
  \plotone{figures/HD106906_lightcurves.pdf}
  \caption{HST/WFC3/IR Lightcurves for HD106906b in the F127M, F139M, and F153M bands. For clarity,  offsets of 5\% and 10\% are applied to the F139M and F153M lightcurves, respectively.}
  \label{fig:lightcurve}
\end{figure}


\subsection{Astrometry}
We follow the procedure detailed in \citet{Bedin2018} for astrometric measurement. Astrometric measurements are made for HD106906 A and b, as well as 25 background stars.
We first measure the raw Cartesian $(x, y)$ coordinates by fitting empirically derived PSFs to the \texttt{flt} images using a software that is adapted from the program \texttt{img2xym\_WFC.09x10}, which is initially developed for ACS/WFC \citep{Anderson2006} and extended for WFC3/IR \citep[e.g., employed in][]{Bedin2018}. The empirical PSFs are from publicly available PSF library released by J. Anderson at STScI \footnote{\url{http://www.stsci.edu/~jayander/WFC3/WFC3IR\_PSFs/}}.  We then apply the most updated geometry correction for WFC3/IR  (derived by J. Anderson and is publicly available \footnote{\url{http://www.stsci.edu/~jayander/WFC3/}}). The corrected Cartesian coordinates within the same epoch are then sigma-clipped averaged,  safely assuming no (sizable) intrinsic motion of sources observed within the same epoch. These procedures result in the geometrically corrected Cartesian coordinates and their uncertainties for each source in each epoch.

We then transform the corrected Cartesian coordinates to the equatorial coordinate system (right ascension, {R.A.}, $\alpha$ and declination, {Dec.}, $\delta$). Common stars with GAIA DR2 astrometry \citep{Gaia2018} are used to find the most general linear transformation (six parameters) that converts $(x, y)$ to $(\xi,\eta)$ (the projections of the equatorial $\alpha$ and $\delta$ coordinates on the tangent plane) and vice versa. $(\xi, \eta)$ are then transformed to $(\alpha, \delta)$ using Equations (3) and (4) in \citet{Bedin2018}.

Considering the non-linearity in $(x, y)$ to $(\alpha, \delta)$ transformation, we adopt a Monte Carlo approach to derive the uncertainties in {R.A.} and {Dec.}  For every source, we generate 1,000 Gaussian distributed samples of $(x, y)$  based on the best-fitting values and their uncertainties. We then transform the Cartesian list  to  a list of {R.A.} and {Dec.} pairs. We calculate the standard deviations of the {R.A.} and {Dec.} as their $1$-$\sigma$ uncertainties. We note that the uncertainties in R.A. and Dec. include PSF-fitting uncertainties but do not include systematic uncertainties that can be introduced by motions of the reference sources that are used to establish the $(x, y)$ to $(\xi,\eta)$ transformation. i.e., the astrometric measurements and uncertainties are accurate with respect to a single epoch, but the uncertainties may be underestimated for comparison of astrometry between two epochs. 

\section{Results}
\subsection{Photometry, Lightcurves, and Variability}

Figure~\ref{fig:lightcurve} shows the corrected and normalized lightcurves in the F127M, F139M, and F153M bands. For single exposures, we achieve average photometric signal-to-noise-ratios (SNR) of  77, 78, and 105 in the F127M, F139M, and F153M bands, respectively. The time-averaged absolute flux intensity\footnote{The conversions are done using the \texttt{PHOTFLAM} values listed in fits file headers.} in these three bands are $6.23\pm0.08\times10^{-13}\fluxunit$, $3.71\pm0.05\times10^{-13}\fluxunit$, and $4.35\pm0.04\times10^{{-13}}\fluxunit$, respectively.
%The flux uncertainties are for average $1-\sigma$ uncertainty in a single image.
For the lightcurves, variations with zero-to-peak amplitude greater than 1\%  are \emph{not} detected in any bands.
The lightcurve features are dominated by random noise for which the primary component is the photon noise.  Relative to flat lines, the three lightcurves have reduced-$\chi^{2}$ of 1.89, 1.47, and 1.1 in the F127M, F139M, and F153M bands, respectively. Only the F127M lightcurve show a trace of temporal variations while the other two lightcurves fully agree with flat lines.

\begin{figure}[!th]
  \centering
  \plotone{figures/HD106906_powerspectrum.pdf}
  \plotone{figures/periodogram_bootstrap}
  \caption{Lomb-Scargle periodogram for the lightcurves of HD106906b. \emph{Upper:} Power spectra for the F127M, F139M, and F153M. The power spectrum for the F127M band lightcurve has a peak at 4.02~hr. The other two power spectra do not have any significant periodicity detection except the high frequency region dominated by random noise. \emph{Lower:} Significance estimate for the 4.02~hr signal. Based on a Bootstrap analysis, the significance of the periodic signal in the F127M band lightcurve  is $2.66\sigma$. The black dashed line shows the power spectrum of the observation window function. We note the window function power spectrum, while has its main peak at 1.60~hr --- HST's orbital period, also has a side lope at 3.92 hr, close to our 4.02~hr periodic signal.}
  \label{fig:periodogram}
\end{figure}

We calculated the Lomb-Scargle power spectra \citep[][Figure~\ref{fig:periodogram}]{Lomb1976} for the lightcurves to investigate lightcurve periodicity. The power spectra for the F139M and F153M lightcurves do not show any significant peaks except in the high-frequency region where the power spectra are dominated by random noise. The lack of signals in the F139M and F153M power spectra is consistent with the featureless lightcurves. The power spectra for the F127M lightcurve has a peak at 4.02~hr. Compared to a flat line, the best-fitting single sine wave with period fixed at 4.02~hr marginally decreases the reduced-$\chi^{2}$ from 1.89 to 1.53. The best-fitting amplitude of the 4.02 hr sine wave is $A = 0.49\pm0.12\%$. Figure~\ref{fig:fold} shows the F127M lightcurve folded to the 4.02 hr period and the best-fitting sine wave. We use a bootstrap method \citep{Manjavacas2017,Zhou2019} to evaluate the significance of the periodogram signal, and show the result in Figure \ref{fig:periodogram}. This analysis yields a $2.66\sigma$ significance of the 4.02 hr periodic signal. The 4.02 hr periodic signal also overlaps with a side-lobe of the periodogram of the observation window functions. The low SNR and the effect from observation window function argue \emph{against} 4.02 hr signal being a robust detection of periodicity in the lightcurve.

In summary, HD106906b only shows marginal  evidence of variability in the F127M band. Lightcurves in the other two bands (water absorption, the red side of water band continuum) are consistent with flat lines.

\begin{figure}[t]
  \centering
  \plotone{figures/F127M_foldedLC.pdf}
  \caption{Phase-folded lightcurve for HD106906b in the F127M band. The lightcurve is folded to a period of 4.02 hr. This period corresponds to the most significant peak in the Lomb-Scargle periodogram. The red line is the best-fitting sine wave.}
  \label{fig:fold}
\end{figure}

\subsection{Spectral Energy Distribution}
Our precise time-averaged photometry, particularly HD\-106906b's flux intensity in the water absorption band is useful for determining fundamental properties, such as \teff{} and \logg{} of HD106906b through spectral energy distribution (SED) fitting.  We combine our photometry with archival photometry to form the SED of HD106906b.  We use HST/ACS/F606W band photometry ($\lambda_{\mathrm{pivot}}=0.596\micron$, FWHM$=0.234\micron$) from \citet{Kalas2015}, $K_{s}$ ($\lambda_{\mathrm{pivot}}=2.145\micron$, FWHM$=0.305\micron$) and $L'$ ($\lambda_{\mathrm{pivot}}=3.774\micron$, FWHM$=0.592\micron$) band photometry from \citet{Bailey2013}. We do not use the archival $J$ band photometry \citep{Wu2016} because our F127M photometry covers similar spectral features and has more than $20\times$ greater SNR. Importantly, our F139M photometry provides a tight $1.4\micron$ water absorption constraint for HD106906b.

We fit the SED of HD106906b to the BT Settl model grid \citep[][]{Allard2012} and present the results in Figure~\ref{fig:SED}. We perform model fitting in AB magnitude scales, which is the form directly available from the model.  We convert observed flux {densities} to AB magnitudes, and bi-linearly (in \teff and \logg dimension) interpolate the model grid (native grid resolution: $\Delta \teff=100\,\mbox{K}, \Delta \log g=0.5$) in magnitude scales. The free parameters are effective temperature $\teff$, surface gravity $\logg$, and scaling parameter $\mathcal{S}$, the ratio between the observed flux over model flux. Because model SEDs are presented in flux {density} at the photosphere surface, the scaling parameter can be transformed to the photospheric radius via $R=\sqrt{\mathcal{S}}\,d$, in which $d$ is the distance of the system. By searching for the minimum $\chi^{2}$, we identify the best-fitting $T_{\mathrm{eff}}=1,800\pm100$~K and $\log g=5.5\pm0.5$.  The scaling parameter corresponds to a radius of  $1.775\pm0.015R_{\mathrm{Jup}}$ at a distance of 103.3~pc \citep{Gaia2018,Gaia2016}. The 1,800~K effective temperature estimate is consistent with previous studies \citep{Bailey2013,Wu2016}, but the surface gravity is not compatible with a low surface gravity assessment. Additionally, the model SED under-predicts the F139M band flux or over-predicts the depth of the water absorption band (Figure \ref{fig:SED}).

\begin{figure*}
  \centering
\plottwo{figures/SEDfit.pdf}{figures/SEDfit_zoomin.pdf}
  \caption{The SED of HD106906 and the best-fitting BT-Settl model. The left panel shows the full observed SED (blue) that includes photometry from both this observation and archival data. The red line and dots are the best-fitting (1800~K, $\log g=5.5$) BT-Settl spectrum \citep[rebinned to $R\sim100$ in a flux conserved manner, ][]{Allard2012} and its corresponding model photometry.  The right panel zooms in the wavelength range for this observation. The orange violin plot shows the  distributions of the best-fitting models. These distributions are drawn from the MCMC fitting posteriors. The flux in the F139M band is significantly under-predicted by the 1800\,K BT-Settl model.}
  \label{fig:SED}
\end{figure*}

\subsection{Astrometry}
\label{sec:astrometry}

In order to establish a precise astrometric reference frame and constrain the relative motion between HD106906b and its host star, we measure the {R.A.} and {Dec.} of 25 sources (BG01 to BG25) that are in the field of view (FoV) of both HST/WFC3 epochs. The average uncertainties in {R.A.}/{Dec.} are 5.3 mas for the 2016 epoch and 2.9 mas for the 2018 epoch, corresponding to 0.041 and 0.023 pixels, respectively. Due to the saturation at the PSF core, HD106906A has one of the {lowest} astrometric precisions of all the sources. Especially in the 2016 epoch, its astrometric uncertainty is 51.2 mas or 0.39 pixel. Astrometric measurements for HD106906 are listed in Table \ref{tab:astrometry} and those for the background sources are listed in Table \ref{tab:bck} in the appendix.

We derive the separations and position angles between HD106906A and b and their uncertainties for the 2016 and 2018 epochs. The separations are $7.11\arcsec\pm0.03\arcsec$ and $7.108\arcsec\pm0.005\arcsec$ in the 2016 and 2018 epochs, respectively. The position angles are $307.5^{\circ}\pm0.3^{\circ}$ and $307.29\pm0.05^{{\circ}}$ in the two epochs, respectively. These separations and position angles are indistinguishable from those measured in the ACS/HRC images \citep{Bailey2013}. Therefore, relative motions between the companion and the star are not detected. The substantial positional uncertainty of HD106906A due to saturation is the bottleneck that limits the astrometric value of these HST images (see \S\ref{sec:HD106906:astrometry-discussion}).

\begin{deluxetable}{llclc}
  
  \tablecaption{HST/WFC3 Astrometry for HD106906 System.\label{tab:astrometry}}
  
  \tablehead{
    \colhead{Object (epoch)} &
    \colhead{{R.A.}} &
    \colhead{$\delta${R.A.}} &
    \colhead{{Dec.}} &
    \colhead{$\delta${Dec.}}\\
    \colhead{} &
    \colhead{[hh mm ss]} &
    \colhead{[mas]} &
    \colhead{[dd mm ss]} &
    \colhead{[mas]} 
  }
  
  \startdata
  HD106906A (2016) & 12 17 53.118 & 16 & $-$55 58 32.136 & 49 \\
  HD106906b (2016) & 12 17 52.444 & 1.1 & $-$55 58 27.8199 & 0.79 \\
  HD106906A (2018) & 12 17 53.108 & 5.6 & $-$55 58 32.158 & 6.7 \\
  HD106906b (2018) & 12 17 52.434 & 2.1 & $-$55 58 27.843 & 2.3 \\
  \enddata
\end{deluxetable}

\subsection{Other Sources in the Field of View}
In order to assess the possible presence of yet undetected companions to HD106906, we construct $33\arcsec\times33\arcsec$ FoV deep images (Figure \ref{fig:bck}) by median-combining the entire HST/WFC3/IR time series for each filter. These images may include yet undiscovered companions of HD106906A. With our observational setup, the water absorption depth can be an effective criterion for selecting candidate ultra-cool objects \citep[e.g.,][]{Fontanive2018}. Here we define the absolute water absorption depth as the difference between the F139M flux density and the average flux {density} in the F127M and F153M bands. We further define the normalized water absorption depth ($\mathcal{D}$) as the absolute depth divided by the average flux density in the F127M and the F153M bands. $\mathcal{D}$ is calculated as

\begin{equation}
  \label{eqn:water}
\mathcal{D} = \frac{(f_{\mathrm{F127M}} + f_{\mathrm{F153M}})/2 - f_{\mathrm{F139M}}}{(f_{\mathrm{F127M}} + f_{\mathrm{F153M}})/2} 
\end{equation}

In all three bands, we calculate the $5\sigma$ contrast curves for contrast-limited point-source detections for the median-combined primary-subtracted images (Figure \ref{fig:contrast_curve}). We construct these contrast curves through a PSF injection-and-recovery process, as detailed in  \citet{Zhou2019}.  We find that the three bands  have almost identical contrast curves, although the F127M image has the deepest the contrast at wide separation.  Our median-combined, primary-subtracted images are sensitive to $\Delta \mbox{mag}=7.7$ at 1\arcsec, $\Delta \mbox{mag}=10.4$ at 2\arcsec, and $\Delta \mbox{mag}=14.2$ at 5\arcsec. Assuming an age of 15 Myr and the evolution tracks of \citet{Saumon2008}, our median-combined, primary-subtracted images can place 5$\sigma$ upper limits for companions more massive than 7\mjup{} at 2\arcsec{} or greater  separations and 2\mjup{} at 4.75\arcsec{} or greater  separations.

\begin{figure}
  \centering
  \plotone{figures/contrast_curve.pdf}
  \caption{Azimuthally averaged contrast curves in F127M, F139M, and F153M for the HD106906 observations.}
  \label{fig:contrast_curve}
\end{figure}

We used the median-combined primary-subtracted images to measure the relative water absorption depth for 25 point sources (from BG01 to BG25, see Table \ref{tab:bck}) that are in the field of view for images taken with both telescope rolls. Figure~\ref{fig:backgroundsources} shows the water absorption depth for each source.  Water absorption is marginally detected in two other sources (BG11 and BG12). Interestingly, these two sources also have the smallest angular separations from HD106906A among all point sources in the field of view. For both BG11 and BG12, their astrometry in the 2016 and 2018 HST/WFC3 observations are consistent within 15~mas and they do not appear to co-move with the HD106906 system. Therefore they are likely background stars.  BG12 {is also very close to HD106906b in angular separation} (0.87\arcsec) and is also in the field of view of the 2004 HST/ACS image.   Based on the HST/ACS/HRC and the HST/WFC3/IR photometry, the SED of this source is best fit by a $3.7\pm0.1\times10^{3}$~K BT-Settl stellar SED model. BG12's fourteen-year baseline astrometry is consistent with that for a stationary background source. Therefore, BG12 is most likely a background K/M giant star. 

\begin{figure}
  \centering
  \plotone{figures/bck_waterdepth.pdf}
  \caption{Measured relative water absorption depths of 25 sources in the field of view. The sources are ranked by their angular distance to HD106906A. Except HD106906b, there are two sources (BG11 and BG12) have water absorptions, but at much weaker levels.}
  \label{fig:backgroundsources}
\end{figure}

\begin{figure*}[!ht]
  \centering
  \plottwo{figures/bckStarDeltaRADEC.pdf}{figures/bckStarSeparation.pdf}
  \caption{Relative astrometry between HD106906b and a background star seen close in projection. Top: The difference in right ascension and declination. Bottom: The separation as a function of time. Past observations of HD106906b are marked as squares.}
  \label{fig:astrometry:bck}
\end{figure*}

We investigate the {apparent trajectory} of background source BG12, {noting that its location at prior epochs could potentially have contaminated observations of HD106906b reported earlier in the literature}.  We calculate  the differences in right ascension ($\Delta${R.A.}),  declination ($\Delta${Dec.}), and the separations between HD106906b and the close background source from the year 2003 (one year before the first direct imaging {reported for} HD106906b) to the year 2023. In this calculation, the close background source is assumed to be stationary and HD106906b is co-moving with its host star at $(\mu_\alpha\cos\delta=-39.01\,\mbox{mas/yr}, \mu_{\delta}=-12.87\,\mbox{mas/yr})$ \citep{Gaia2016, Gaia2018}. The results are shown in Figure~\ref{fig:astrometry:bck}. In the same figure, we also marked the expected positions of the close companion in previous observations \citep{Bailey2013, Wu2016, Lagrange2016, Daemgen2017} assuming BG12 is a stationary background star.

Figure~\ref{fig:astrometry:bck} demonstrates that HD106906b, due to its proper motion, has been approaching  -- in projection -- to {the location of BG12}  over the years. The separation between HD1006906b and {BG12} has been decreasing from 1\arcsec.29 (2004, first available image) to 0\arcsec.87 (this study). In the study of \citep{Bailey2013, Wu2016, Daemgen2017},  HD106906b should have a separation of 0.95\arcsec{}-1.05\arcsec to {BG12}, assuming it is stationary. It is unlikely that {BG12} contaminated those measurements, because the separations in those observation epochs were significantly greater than the spatial resolutions of those observations. With the brightness contrast of the two objects, in the worst case, the contamination of the background star to HD106906b's broadband photometry is  $<7.5\%$ in the near-infrared. 



% \subsection{Astrometry -- relative to the background star}
% \label{sec:astrometry}



% \Interestingly{Large scale structures}


\section{Discussion}
\subsection{Rotational modulations of HD106906b}

\begin{figure*}[!t]
  \centering
    \plotone{figures/LS_Comparison}
  \caption{Comparison of the periodograms between those for the ten brightest background stars and those for HD106906b. The two background stars (BG06 in the F127M band and BG03 in the F139M band) that show similar signals to HD109606b's in their periodograms do not show significant variations in the folded lightcurves.}
  \label{fig:all-periodograms}
\end{figure*}

We evaluate the modulation significance in HD106906b's observed lightcurve from both instrumental and astrophysical perspectives. From the instrumental point of view, we have two arguments against the possibility that the modulation signal that we observe in HD106906b's F127M lightcurve arises from instrumental systematics. First, the F127M, F139M, and F153M observations were taken \emph{de facto} contemporaneously with identical instrument set-ups except the choice of filters. Systematics that introduce periodic/sinusoidal signals at 4 hr timescale in the F127M lightcurve should have a similar effect on the other two lightcurves. The agreement of the  F139M and F153M lightcurves with flat lines is inconsistent with the possibility that modulations of the F127M lightcurve {are} due to systematics. Second, similar modulations do not appear in the lightcurves of any of the 20 background stars in the same images. We measure and analyze lightcurves of ten brightest background stars (BG01 to BG10) that are in the field of view of both telescope roll angles and are not affected by the diffraction spikes of the primary PSFs.
Figure \ref{fig:all-periodograms} shows the comparison between the periodograms of the F127M, F139M, and F153M lightcurves of the background stars and that for F127M lightcurve of HD106906b. Most periodograms of the background star do not show significant signals with similar periodicity to HD106906b except two objects (BG03 and BG06). However, when we fold the lightcurves of those two objects to the periods of the corresponding peaks in the periodograms, the folded lightcurves are consistent with flat lines. These two lines of evidence argue that systematics are unlikely to be the cause of the modulations observed in HD106906b's F127M lightcurve.

From the astrophysical perspective, we can qualitatively evaluate the likelihood for HD106906b, \added{an early L-type planetary-mass companion} to be rotationally modulated only in the F127M band but not in the other two bands. Multi-wavelength and time-resolved observations of ultra-cool dwarfs have found that the rotational modulations for the majority  brown dwarfs and planetary-mass companions are wavelength-dependent and have higher amplitudes at shorter wavelengths than longer wavelengths \citep[e.g.,][]{Apai2013,Yang2015,Zhou2016,Schlawin2017,Zhou2019}. These {findings} are consistent with a model prediction based on Mie-scattering calculation \citep{Hiranaka2016, Lew2016,Schlawin2017}. Additionally, the 1.4 \micron{} water absorption or the F139M band sometimes show reduced rotational modulation amplitude \citep[e.g.,][]{Apai2013}, due to water vapor opacity elevating the photosphere at this wavelength.  Therefore, rotational modulations only appearing in the band with the shortest wavelength of our observation is qualitatively consistent with model predictions and previous observations, particularly those for planetary-mass companions \citep{Zhou2016,Zhou2019}. If we assume that the wavelength dependence of HD106906b's rotational modulations is the same as that measured in 2M1207b \citep{Zhou2016} as 2M1207b (a mid-L-type planetary-mass companion) is HD106906b's close analog that also has modulation detected, we would expect the modulation amplitude in the F153M band to be 0.6\%. Our observation is not sensitive to such small amplitude modulations. Therefore, if the overall modulation amplitude is low, it is likely that the signal is only detected in the bluest band of the observation, which is consistent with our observations.

These two lines of evidence support the interpretation that the modulations we see in HD106906b's F127M lightcurve are astrophysical and, in particular, caused by heterogeneous clouds. Nevertheless, we emphasize that the amplitude of the signal is marginal. Our evaluation of the rotational modulation and rotation period for HD106906b remain \emph{tentative}.


\subsection{The Spectral Energy Distribution of  HD106906b}

\begin{figure*}
  \centering
  \plottwo{figures/logg_water.pdf}{figures/Teff_water.pdf}
  \caption{1.4 \micron{} water band depths predicted by the BT Settl model. Left: With a fixed $T_{\mathrm{eff}}=1800$ K, models with $\log g$ between 3 to 5.5 all over-predicts the water band depths. Right: With a fixed $\log g=5.5$, it requires $T_{\mathrm{eff}}=2500$ K for the model to match the observation.}
  \label{fig:waterdepth}
\end{figure*}
Two issues have emerged in our SED fitting results. First, the best-fitting model over-predicts the strength of  water absorption at 1.4\,\micron{}. Second, the best-fitting surface gravity (on the BT-Settl grid, with existing photometry) is inconsistent with previous low-gravity assessments for HD106906b nor the surface gravity expected for a young planetary-mass object.

To investigate the first issue, we calculate the water band absorption depths defined in Equation \ref{eqn:water} in the BT-Settl model as functions of $T_{\mathrm{eff}}$ and $\log g$ and compare them with the {observationally determined} value. As shown in Figure \ref{fig:waterdepth}, for a fixed $T_{\mathrm{eff}}$ of 1800 K, all BT Settl models regardless of \logg{} over-predict the strength of water absorption. For a fixed $\log g$ of 5.5, to match the observed water band depth, $T_{\mathrm{eff}}$ needs to be raised to $2,500$ K, although such a high temperature is {inconsistent} with the overall SED shape. Therefore, the mismatch between the HD106906b's observed and the model SEDs in the 1.4 \micron{} water bands demonstrates the BT-Settl model may not capture all the physical/chemical processes that are important in HD106906b.

One potential solution that can improve the unsatisfying water band absorption fit is to introduce composite spectra. Linear combinations of model spectra often fit better to observed ultra-cool dwarf spectra than single-component models \citep[e.g.,][]{Marley2010,Apai2013}. \replaced{In the case of HD106906b, a combination of $\teff\sim$~2,600~K and $\teff\sim$~1,800~K model spectra will reduce the water band depth in the model and have a better fit to the observation, although due to the limitation of number of photometric data point our composite model fit does not converge to {a single} best-fit solution.}{Combining the 1,800\,K model spectrum with a higher temperature ($>2,600\,\mathrm{K}$) model will reduce the water absorption band depth and alleviate the model-observation conflict in the WFC3/IR bandpass. However, we note that a $>800\,K$ temperature contrast in an ultra-cool atmosphere is unprecedented even for brown dwarfs with large amplitude rotational modulations. Additionally, such a composite spectrum will cause imperfect SED fit at longer wavelength (K and L' bands).}  A composite model is a piece of evidence for the heterogeneity of its atmosphere. {The lack of large-amplitude rotational modulation in the HD 106906b lightcurve might be indicative of a (nearly) pole-on geometery of its rotational axis \citep[e.g.,][]{Vos2017b}. }
This prediction can be tested by $v\sin i$ measurements from high-resolution spectroscopic observations \citep[e.g.,][]{Snellen2014,Vos2017b,Bryan2018}.

The surface gravity of HD106906b has been discussed in \citet{Bailey2013} and \citet{Daemgen2017}.  Based on its ``triangle-shaped'' $H$ band spectrum, both studies classified HD106906b as a low-to-intermediate surface gravity ($\logg\sim4\mbox{-}4.5$) object. The equivalent widths of the K I absorption lines measured in \citet{Daemgen2017} are also consistent with an intermediate surface gravity classification. In addition, low surface gravity is also consistent with the young and low mass nature of HD106906b. However, our SED fitting yields a very high gravity ($\logg=5.5$) result. With a fixed $T_{\mathrm{eff}}$ of 1800 K, the lower surface gravity models significantly over-predict the F153M and $K_{\mathrm{s}}$ band flux {densities}. Therefore, they are not preferred by the SED fitting. This inconsistency demonstrates the challenges posed by young, planetary-mass ultracool atmospheres.



\subsection{Astrometric Constraints on the  HD106906 System}
\label{sec:HD106906:astrometry-discussion}
With a temporal baseline of 14 years, three epochs of HST observations are not able to detect relative motion between HD106906b and its host star. Assuming a face-on, circular orbit and an orbital radius of 732 au, we expect an orbital arc length for HD106906b to be 37.1~mas in 14 yr (first epoch with ACS in 2004) or 5~mas in 2~yr (between the two WFC3 epochs). These arc lengths correspond to $12.8\times$ and $1.72\times$ the average $1\sigma$ astrometric uncertainty in the 2018 epoch. As a result, the HST images could resolve these orbital motion if their precisions are not limited by saturation.

Astrometric constraints of the HD106906 system are critical to studying the system's formation and dynamical evolution history \citep[e.g., ][]{DeRosa2019} and for measuring the dynamical mass of the planet \citep[e.g.,][]{Snellen2018,Dupuy2019}. The design of future observations should consider optimization for astrometric precisions, which includes avoiding saturation, increasing spatial resolution through dithering, repeating at the same celestial orientation angles and re-use of the same guide stars. In the $33\times33$ arcsec$^{2}$ FoV of WFC3 images, there are seven background sources that have {celestial} coordinates and proper motion measurements from GAIA DR2. Using these sources to register the WFC3 image with GAIA can calibrate the absolute astrometry to sub-mas precision level \citep{Bedin2018}. Future astrometric analysis of HD106906 system will  benefit from our background source catalog (Table \ref{tab:bck}).

\section{Summary and Conclusions}

\begin{enumerate}
\item We observed the planetary-mass companion HD106906b with seven consecutive HST orbits in HST WFC3/IR's direct-imaging mode. Applying two-roll differential imaging and PSF-fitting photometry, we have achieved single-frame photometric precisions of 1.3\%, 1.3\% and 0.9\% in the lightcurves in the F127M, F139M, and F153M bands, respectively. The F127M light curve presents a tentative ($2.66\sigma$) variability signal that best-fit to a $P=4.02$\,hr 

\item The marginal detection of the F127M band modulation and the non-detections in the other two bands are consistent with the wavelength dependence of modulation amplitudes previously observed in other brown dwarfs and planetary-mass companions. The wavelength-dependent rotational modulations are also consistent with the interpretation that the modulations are caused by heterogeneous condensate clouds. The marginally-detected, low-amplitude modulations agree with the expectation that early-L type dwarfs are less likely to be have large-amplitude variability compared to the L/T transition types \citep[e.g.][]{Radigan2014,Metchev2015}.

\item Our observations provide precise  photometry for HD106906b in the HST/WFC3/IR F127M, F139M, and F153M bands. This is also the first precision photometric measurement for HD106906b in the 1.4\,\micron{} water absorption band. We combine our three bands of photometry with archival data to form an SED for HD106906b and perform SED model fitting on the BT-Settl model grid. We find a best-fitting effective temperature of 1800~K, consistent with literature results, and a best-fitting surface gravity $\log g$ of 5.5,  significantly higher than previous estimates and inconsistent with HD106906b being a young and planetary mass object. Also, the observed F139M band flux intensity for HD106906b is significantly higher than the best-fitting model value.
  %\add{The BT-Settl models systematically over-estimate the 1.4 \micron{} water absorption depth, leading to the incorrectly high \logg{} value.}
  These inconsistencies suggest the challenges in modeling the atmospheres of young planetary-mass objects.

\item We combine WFC3/IR images to form primary-subtracted deep images and search for planetary-mass companions in the field of view. Our composite images are sensitive to planets with masses down to $< 2 \mjup$. We used measurements of the 1.4\micron{} water absorption to {arbitrate between close companion candidates and background stars} (i.e., substellar companions should show significant water absorptions). We did not discover  any new companions. We did find two point sources that have lower fluxes in the F139M band. However, both sources do not appear to co-move with the HD106906 system. One of the two objects is in close vicinity to HD106906b (0.85\arcsec angular separation). Based on its astrometry and SED fitting results, this object is likely a background K/M giant star.  Based on GAIA DR2 astrometry and proper motion, the angular distance between HD106906b and this background star is decreasing and will be on the level of 0.7\arcsec to 0.8\arcsec{}  in the 2020s. Future observations of HD106906b will need to  carefully eliminate the flux contamination from this background star.

\item We measured astrometry for HD106906A and b, as well as for the background sources. The separations and position angles between HD106906A and b in the 2016 and 2018 epochs WFC3 images do not deviate from  those in the 2004 ACS/HRC images for more than 1$\sigma$ uncertainty. The saturated PSF core of  HD106906A limits our sensitivity in probing the relative motion between HD106906A and b. HST/WFC3 observations that avoid saturating the primary will at least place strong constraint on whether HD106906b is on a face-on circular orbit and may even resolve the planet's orbital motions.
\end{enumerate}


\bibliographystyle{yahapj}
\bibliography{library}

\appendix
\section{Background source information}

The sky locations for background sources are illustrated in Figure \ref{fig:bck}. Table \ref{tab:bck} summarizes the information for background sources.

\begin{figure}[h]
  \centering
  \includegraphics[width=0.8\textwidth]{figures/backgroundStar.png}
  \caption{Illustration of sky locations of background sources in the FoV.}
  \label{fig:bck}
\end{figure}

\begin{rotatetable*}
\begin{deluxetable*}{cccccccc}
  \footnotesize
   \tablecaption{Background Sources in the Field of View\label{tab:bck}}
  \tablehead{
    \colhead{Source ID} &
    \colhead{RA} &
    \colhead{Dec} &
    \colhead{RA} &
    \colhead{Dec} &
    \colhead{Flux$_{\mathrm{F127M}}$} &
    \colhead{Flux$_{\mathrm{F139M}}$} &
    \colhead{Flux$_{\mathrm{F153M}}$} \\
    \colhead{} &
    \colhead{2016.08} &
    \colhead{2016.08} &
    \colhead{2018.43} &
    \colhead{2018.43} &
    \colhead{\fluxunit} &
    \colhead{\fluxunit} &
    \colhead{\fluxunit} 
}
    \startdata
    % HD106906A & 12h17m53.11855s & -55d58m32.1356s & 12h17m53.10799s & -55d58m32.1575s \\
    % HD106906b & 12h17m52.44400s & -55d58m27.8199s & 12h17m52.43412s & -55d58m27.8430s \\
      BG01    & 12h17m51.08033s & -55d58m32.8722s & 12h17m51.08012s & -55d58m32.8703s    & 9.31e-20 & 9.16e-20 & 7.58e-20 \\
      BG02    & 12h17m52.53287s & -55d58m11.6989s & 12h17m52.53285s & -55d58m11.7027s & 9.33e-20 & 9.18e-20 & 7.60e-20 \\
      BG03    & 12h17m52.19115s & -55d58m23.9812s & 12h17m52.19254s & -55d58m23.9863s  & 9.33e-20 & 9.18e-20 & 7.60e-20 \\
      BG04    & 12h17m53.48526s & -55d58m13.6424s & 12h17m53.48539s & -55d58m13.6401s & 9.29e-20 & 9.17e-20 & 7.58e-20 \\
      BG05    & 12h17m53.33002s & -55d58m18.7354s & 12h17m53.32989s & -55d58m18.7321s & 9.33e-20 & 9.18e-20 & 7.61e-20 \\
      BG06    & 12h17m52.83135s & -55d58m17.0499s & 12h17m52.83170s & -55d58m17.0504s & 9.30e-20 & 9.16e-20 & 7.60e-20 \\
      BG07    & 12h17m52.53095s & -55d58m25.2858s & 12h17m52.52897s & -55d58m25.2814s & 9.34e-20 & 9.16e-20 & 7.60e-20 \\
      BG08    & 12h17m51.86552s & -55d58m30.5719s & 12h17m51.86637s & -55d58m30.5734s & 9.61e-20 & 9.19e-20 & 7.30e-20 \\
      BG09    & 12h17m51.20143s & -55d58m19.6506s & 12h17m51.20233s & -55d58m19.6548s  & 9.32e-20 & 9.20e-20 & 7.57e-20 \\
      BG10    & 12h17m52.44974s & -55d58m23.4974s & 12h17m52.44988s & -55d58m23.4991s  & 9.36e-20 & 9.17e-20 & 7.62e-20 \\
      BG11    & 12h17m53.27647s & -55d58m25.8212s & 12h17m53.27557s & -55d58m25.8227s & 9.33e-20 & 9.23e-20 & 1.16e-19 \\
      BG12    & 12h17m52.40001s & -55d58m28.6510s & 12h17m52.40096s & -55d58m28.6424s  & 7.74e-20 & 8.17e-20 & 2.16e-19 \\
      BG13    & 12h17m54.31592s & -55d58m26.2282s & 12h17m54.31764s & -55d58m26.2231s  & 9.30e-20 & 9.20e-20 & 7.67e-20 \\
      BG14    & 12h17m53.57930s & -55d58m18.5369s & 12h17m53.57942s & -55d58m18.5387s  & 9.02e-20 & 9.19e-20 & 7.59e-20 \\
      BG15    & 12h17m53.28572s & -55d58m19.4339s & 12h17m53.28540s & -55d58m19.4317s & 7.89e-20 & 8.43e-20 & 9.86e-20 \\
      BG16    & 12h17m53.23686s & -55d58m13.4621s & 12h17m53.23667s & -55d58m13.4615s & 8.73e-20 & 8.82e-20 & 7.61e-20 \\
      BG17    & 12h17m51.94942s & -55d58m16.2192s & 12h17m51.94916s & -55d58m16.2213s & 8.72e-20 & 9.10e-20 & 7.64e-20 \\
      BG18    & 12h17m52.35709s & -55d58m16.4143s & 12h17m52.35722s & -55d58m16.4142s  & 8.78e-20 & 8.92e-20 & 7.58e-20 \\
      BG19    & 12h17m50.95989s & -55d58m35.9010s & 12h17m50.95988s & -55d58m35.9000s  & 8.00e-20 & 8.66e-20 & 7.42e-20 \\
      BG20    & 12h17m51.68586s & -55d58m30.8703s & 12h17m51.68641s & -55d58m30.8587s & 2.69e-19 & 3.18e-19 & 7.21e-20 \\
      BG21    & 12h17m51.50254s & -55d58m32.1091s & 12h17m51.50420s & -55d58m32.1134s & 8.23e-20 & 8.10e-20 & 7.09e-20 \\
      BG22    & 12h17m52.15373s & -55d58m19.3199s & 12h17m52.15175s & -55d58m19.3031s & 8.66e-20 & 9.08e-20 & 7.64e-20 \\
      BG23    & 12h17m54.01685s & -55d58m19.5122s & 12h17m54.01515s & -55d58m19.5069s  & 9.41e-20 & 9.14e-20 & 7.48e-20 \\
      BG24    & 12h17m51.54886s & -55d58m39.3202s & 12h17m51.54881s & -55d58m39.3169s & 8.68e-20 & 8.45e-20 & 7.29e-20 \\
      BG25    & 12h17m52.15398s & -55d58m40.9675s & 12h17m52.15317s & -55d58m40.9600s & 9.32e-20 & 9.18e-20 & 7.46e-20 
      \enddata
\end{deluxetable*}
  \end{rotatetable*}

\end{document}

%%% Local Variables:
%%% mode: latex
%%% TeX-master: t
%%% End:

% LocalWords:  AAS
